\documentclass[]{article}

\usepackage{pythontex}

\usepackage{graphicx}
\usepackage{pgf}
\usepackage{float}
\usepackage[margin=3cm,noheadfoot]{geometry}

%opening
\title{An Example of Astropy and PythonTeX Integration}
\author{Stuart Mumford}

\begin{pythontexcustomcode}{py}
import sys

import numpy as np

import matplotlib

matplotlib.use('pgf')

matplotlib.rc('text', usetex=True)
matplotlib.rc('font', family='serif')
matplotlib.rc('font', size=14.0)
matplotlib.rc('font', weight='normal')-
matplotlib.rc('legend', fontsize=14.0)
matplotlib.rc("pgf", texsystem="pdflatex")

import matplotlib.pyplot as plt
from mpl_toolkits.axes_grid1 import make_axes_locatable

def repr_latex_str(thing):
    if hasattr(thing, '_repr_latex_'):
        return thing._repr_latex()
    else:
        # do default thing
        pass
    

\end{pythontexcustomcode}

\begin{document}

\maketitle

\section{Basic PythonTeX}

The basic PythonTeX environments are \verb|\begin{pycode}| and \verb|\py|. \verb|pycode| is a block environment for including Python code, \verb|\py| is a inline output environment, for displaying output.

A simple example of both is to create a variable in a \verb|pycode| environment and then inline it in the text using \verb|\py|:

\begin{pyverbatim}
\begin{pycode}
import astropy.units as u

myquantity = 100 * u.kg/u.m**2
\end{pycode}
\end{pyverbatim}

\begin{pycode}
import astropy.units as u

myquantity = 100 * u.kg/u.m**2
\end{pycode}

You can then include this variable in a sentence using \verb~\py|myquantity|~. The density of Astropy is \py|myquantity|. Quantity has a little feature which is useful here: \verb~\py|myquantity_repr_latex_()|~ gives \py|myquantity._repr_latex_()|.

\section{Configuring PythonTeX}

It is possible to execute python code in PythonTeX that affects all environments throughout the document. This is commonly used for global imports and configuration.
It can be achieved using the \verb|\begin{pythontexcustomcode}| environment.
This block is especially useful for configuring matplotlib to use settings compatible with PythonTeX.
The matplotlib pgf backend can be configured to use LaTeX for typesetting, and the fonts configured to match that of your LaTeX document.
An example of this configuration is shown below, and used in this document.

\begin{pyverbatim}
\begin{pythontexcustomcode}{py}
import sys

import numpy as np

import matplotlib
matplotlib.use('pgf')

matplotlib.rc('text', usetex=True)
matplotlib.rc('font', family='serif')
matplotlib.rc('font', size=14.0)
matplotlib.rc('font', weight='normal')
matplotlib.rc('legend', fontsize=14.0)
matplotlib.rc("pgf", texsystem="pdflatex")
import matplotlib.pyplot as plt
from mpl_toolkits.axes_grid1 import make_axes_locatable


\end{pythontexcustomcode}
\end{pyverbatim}

This configuration can be used to generate a plot:

\begin{pyblock}
from sunpy.data.sample import AIA_171_IMAGE
import sunpy.map

aiamap = sunpy.map.Map(AIA_171_IMAGE)

fig = plt.figure()
im = aiamap.plot()

fig.savefig('myplot.pdf')
\end{pyblock}

which can be included with the normal LaTeX figure environment:\footnote{The pgf package is required.}

\begin{verbatim}
\begin{figure}[H]
	\pgfimage[width=\columnwidth]{myplot}
\end{figure}
\end{verbatim}

\begin{figure}[H]
	\pgfimage[width=0.7\columnwidth]{myplot}
\end{figure}

\section{Astropy Table}

We can make use of this integration between LaTeX and Python to make the use of tools with LaTeX representation support built in even easier.
The Astropy Table class has a very powerful TeX formatter, which we will use to inline a Table in this document.

\begin{pyblock}
import StringIO
from astropy.table import Table

mytable = StringIO.StringIO()

t = Table.read("ftp://cdsarc.u-strasbg.fr/pub/cats/VII/253/snrs.dat",
	           readme="ftp://cdsarc.u-strasbg.fr/pub/cats/VII/253/ReadMe",
	           format="ascii.cds")

# Crop the table down a bit
t = t[t.colnames[0:10]][0:50]

t.write(mytable, format='latex', latexdict={'tablealign':'H'})

\end{pyblock}

\py|mytable.getvalue()|


\section{Figure Manager}

I pulled this out of my Thesis repo this morning!!

\begin{pycode}[chapter2]
from texfigure import Chapter
ch2 = Chapter(pytex, 2, './')
\end{pycode}

So, as I pointed out yesterday, SunPy is awesome! It makes graphs and everything.

\begin{pycode}[chapter2]
from sunpy.data.sample import AIA_171_IMAGE
import sunpy.map

aiamap = sunpy.map.Map(AIA_171_IMAGE)

fig = plt.figure(figsize=(10,10))
im = aiamap.plot()
aiamap.draw_grid()


ch2.save_figure('sunpy1', fig)
\end{pycode}

\py[chapter2]|ch2.build_figure('sunpy1',
width=r"0.9\columnwidth",
caption=r"EUV image of the million degree solar corona, taken with the Atmospheric Imaging Assembly on the NASA Solar Dynamics Observatory.")|

You can now reference the figure above, \ref{fig:sunpy1}.

\end{document}
